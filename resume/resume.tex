%-------------------------------------------------------------------------------
\documentclass[UTF8,AutoFakeBold]{resume}
%-------------------------------------------------------------------------------
\usepackage{zh_CN-Adobefonts_external}
\usepackage{linespacing_fix}
\usepackage{cite}
\usepackage{float}
\usepackage{fontspec}
\usepackage{graphicx}
\usepackage{wrapfig} 
%-------------------------------------------------------------------------------
\usepackage{amsmath,bm}
%-------------------------------------------------------------------------------
\usepackage{booktabs}
\usepackage{hyperref}
%-------------------------------------------------------------------------------
%-------------------------------------------------------------------------------
\begin{document}
\pagenumbering{gobble}
%-------------------------------------------------------------------------------
    \begin{figure}[h]
        \flushright
        \includegraphics[height=3.0cm,width=2.5cm]{CV.png}
    \end{figure}
% \vspace{-0.175\linewidth}
\vspace{-0.14\linewidth}
%-------------------------------------------------------------------------------
%%%%%%% \vspace{-0.175\linewidth}   %% 标注考研成绩 -0.175\linewidth        %%%%%%
%%%%%%% \vspace{-0.15\linewidth}    %% 隐藏考研成绩 -0.15\linewidth         %%%%%%
%-------------------------------------------------------------------------------
    \begin{minipage}[t]{0.175\textwidth}
        \centering
        \LARGE\fangsong\textbf{{梁天一}}
    \end{minipage}
%-------------------------------------------------------------------------------
%%%%%%% 取消以下注释,标注考研成绩
%-------------------------------------------------------------------------------
\hspace{1em}
    % \begin{minipage}[t]{0.4\textwidth}\kaishu\textbf{
    %     \centering
    %         \begin{tabular}{ccccc}
    %             \toprule[1.5pt]
    %             思想政治理论&英语(一)&专业课一&专业课二&初试总分\\
    %             \midrule[1pt]
    %             80&80&130&130&420\\
    %             \bottomrule[1.5pt]
    %         \end{tabular}
    %     }
    % \end{minipage}
\vspace{1.5mm}
\par
\hspace{1em}
\faMapMarker \hspace{0.25em}\fangsong\textbf{上海\textbullet 普陀区}
\hspace{0.25em}{\small{\textbullet}}\hspace{0.25em}
% \makebox[0.75em][c]{\faMars}
% \hspace{0.25em}{\small{\textbullet}}\hspace{0.25em}
\fangsong\textbf{26岁}
\hspace{0.25em}{\small{\textbullet}}\hspace{0.25em}
\faPhone \hspace{0.25em}\textbf{(+86)189-6490-7890}
\hspace{0.25em}{\small{\textbullet}}\hspace{0.25em}
\faEnvelope \hspace{0.25em}\textbf{\textit{tianyilt@qq.com}}
% \faGithub \hspace{0.25em}\textbf{\href{https://github.com/tianyilt}{\textit{https://github.com/tianyilt}}}
\hspace{0.25em}\textbf{\href{https://github.com/tianyilt}{\textit{tianyilt}}}
\vspace{1mm}
\vspace{1mm}
%-------------------------------------------------------------------------------
%%%%%%% 取消以下注释,隐藏考研成绩
%-------------------------------------------------------------------------------
% \hspace{0.5em}\fangsong\textbf{\makebox[0.5em][c]{\faVenus}}
% \hspace{0.25em}{\small{\textbullet}}\hspace{0.25em}
% \fangsong\textbf{22岁}
% \hspace{0.25em}{\small{\textbullet}}\hspace{0.25em}
% \faPhone \textbf{\hspace{0.2em}(+86)188-8888-8888}
% \hspace{0.25em}{\small{\textbullet}}\hspace{0.25em}
% \faEnvelope \hspace{0.25em}\textbf{\textit{Oliviabcdef@163.com}}
% \vspace{12mm}

%-------------------------------------------------------------------------------
%-------------------------------------------------------------------------------
\section{\hspace{0.25em}\makebox[0.75em][c]{\faGraduationCap} \fangsong\textbf{教育背景}}
% \datedsubsection{\textbf{华东师范大学(双一流)}\quad \textbf{上海智能教育研究院}\quad \textbf{Ph.D.}}
% {\quad \textbf{2024.9 \textasciitilde \ 2028.7}}
\datedsubsection{\textbf{华东师范大学(双一流)}\quad \textbf{计算机科学与技术}\quad \textbf{学术型硕士}}
{\quad \textbf{2021.9 \textasciitilde \ 2024.7}}
    \begin{minipage}[t]{0.40\textwidth}
    	\begin{itemize}
    		\item \kaishu\textbf{一作论文: }3篇CCF-A在投
    	\end{itemize}
    \end{minipage}
    \begin{minipage}[t]{0.20\textwidth}
    	\begin{itemize}
    		\item \kaishu\textbf{专利: }2篇
    	\end{itemize}
    \end{minipage}
    \begin{minipage}[t]{0.40\textwidth}
    	\begin{itemize}
    		\item \kaishu\textbf{平均学分成绩: }91.7/100.0
    	\end{itemize}
    \end{minipage}

    \begin{itemize}
      \item \kaishu\textbf{研究方向: }LLM 跨模态 Agents, 计算机图形学 3DAIGC, 多模态编辑
    \end{itemize}
\datedsubsection{\textbf{华东理工大学(双一流学科)}\quad \textbf{数学与应用数学\&计算机科学与技术}\quad \textbf{本科}}
{\quad \textbf{2017.9 \textasciitilde \ 2021.7}}
    \begin{minipage}[t]{0.30\textwidth}
    	\begin{itemize}
    		\item \kaishu\textbf{国家奖学金\&上海市奖学金}
    	\end{itemize}
    \end{minipage}
    \begin{minipage}[t]{0.25\textwidth}
    	\begin{itemize}
    		\item \kaishu\textbf{平均学分绩点: }3.73/4.0
    	\end{itemize}
    \end{minipage}
    \begin{minipage}[t]{0.25\textwidth}
    	\begin{itemize}
    		\item \kaishu\textbf{专业排名: }2/97(2\%)
    	\end{itemize}
    \end{minipage}
    %     \begin{minipage}[t]{0.40\textwidth}
    % 	\begin{itemize}
    % 		\item \kaishu\textbf{研究方向: }LLM 跨模态 Agents, 计算机图形学, 生成式图像编辑
    % 	\end{itemize}
    % \end{minipage}
%-------------------------------------------------------------------------------

%-------------------------------------------------------------------------------
\section{\hspace{0.25em}\makebox[0.75em][c]{\faUsers} \fangsong\textbf{实习经历}}

\datedsubsection{\textbf{上海炎阳萌狮教育科技有限公司}\quad \small{\faMapMarker \fangsong\textbf{上海}}}
{\large\textbf{2023.5 \textasciitilde \ 2023.12}}
\role {研发部门}{技术合伙人}
    \begin{itemize}
      % \item \kaishu 规划和落实技术路线。基于飞书制定团队工作流。
      % \item \kaishu 对接公司与华东师范大学计算机学院, 参与初创项目的包装, 通过答辩获得大学生创业雏鹰计划基金资助, 觉群文创玉佛寺基金资助。 
      \item \kaishu 难点: 大模型做青少年科创存在幻觉问题, 当前工具对用户使用门槛高。解决方法:算法上开发基于RAG的青创大模型来消除幻觉; 设计能够协调检索摘要写报告的多智能体把用户从繁重的检索中解脱, 并以用户易于理解的方式呈现检索结果。开发环境项目网站:\href{http://106.14.61.143:8889/ai_chat}{i-lion}。
    \end{itemize}
%-------------------------------------------------------------------------------

\datedsubsection{\textbf{深兰科技 深兰研究院}\quad \small{\faMapMarker \fangsong\textbf{上海}}}
{\large\textbf{2019.7 \textasciitilde \ 2019.8}}
\role {生成式算法工程师}{实习生}
    \begin{itemize}
      \item \kaishu 难点:电视台项目需求开放性强。解决方法:用\href{https://zhuanlan.zhihu.com/p/671425463}{工具链}检索任务中SOTA开源项目,Dotfile+Docker部署测试开源项目。解决视频预处理问题。大量测试和改进的工程包括图像生成超分辨率去噪去模糊着色三维人体重建三维姿态估计等。容器化方法被我迁移到\href{https://ecnuvis.net/}{当前课题组}的\href{https://zhuanlan.zhihu.com/p/682477952}{深度学习容器化解决方案}。
      \item \kaishu 难点:视频动画化不能实时。解决方法, 变成生产者消费者模型, 用Tornado框架构建Buffer异步请求,把视频帧取模分配给多GPU,把fps从5加速到30。
      % \item \kaishu 
    \end{itemize}
%-------------------------------------------------------------------------------

% \datedsubsection{\textbf{哔哩哔哩(上海)有限公司}\quad \small{\faMapMarker \fangsong\textbf{上海}}}
% {\large\textbf{2022.9 \textasciitilde \ 2022.10}}
% \role {哔哩吧啦部门}{实习生}
%     \begin{itemize}
%       \item \kaishu 负责哔哩吧啦;负责哔哩吧啦;负责哔哩吧啦
%     \end{itemize}
%-------------------------------------------------------------------------------
%-------------------------------------------------------------------------------
\section{\hspace{0.25em}\makebox[0.75em][c]{\faFlask} \fangsong\textbf{科研项目}}
\datedsubsection{\textbf{PromptPlanewalker: Interactive Visual Prompt Engineering \linebreak for Text-to-3D Generation} \ 会议论文VIS2024(在投) \qquad 第一作者}{\textbf{2023.2\textasciitilde \ 2024.3}}
    \begin{itemize}
    \item[\faThumbTack] \kaishu 任务创新: 文生3D视觉提示工程,该任务侧重于对生成的3D模型的多视图图像进行跨模态评估,以帮助用户选择满意的结果并迭代改进后续提示。
    \item[\faThumbTack] \kaishu 可视化创新:有效表示提示文本, 3D物体和多维度评分关系的三种D3写的视觉表示。包括提示词推荐词云, 挑结果神器多视图卫星图,以及跨注意力的多视图热图。
    \item[\faThumbTack] \kaishu 系统创新: 利用多视图图像作为评分媒介,通过模型生成和检索来得到3D候选后。多视图混合评分函数从多个角度符合人类直觉地评估3D模型的语义和视觉质量。成果可以直接用于跨模态的vlm打分任务, 用低阶打分消除幻觉,然后llm进行高阶打分对结果进行解释。
        % \item[\faThumbTack] \kaishu 改进了文生3D工具,对输入提示词进行联想,然后使用双分支进行模型生成和使用3D预训练模型进行检索,得到3D候选, 并且用VLM进行多视图评分。提出了Treemap 词云和多视图卫星图等功能,用于快速选择和可视化候选模型。注意力引导的热图可视化了提示词的关键词对生成的模型的不同视图的贡献。
        % \item[\faThumbTack] \kaishu 针对存在问题快速架构问题解决方案, 积极沟通完成论文规划, 然后独立完成FASTAPI后端,部署了5种3D生成模型,包括TripoSR等, 然后用CLIP attention带来局部可解释性以及评分和联想功能的提示词模板。协作完善前端与论文。
        % \item[\faThumbTack] \kaishu 成果可以直接用于跨模态的llm评价, 用低阶打分消除幻觉,然后llm进行高阶打分对结果进行解释。
    \end{itemize}
%-------------------------------------------------------------------------------
\datedsubsection{\textbf{TextCenGen: Attention-Guided Text-Centric Background Adaptation \linebreak for Text-to-Image Generation} \ 会议论文IJCAI2024(在投) \qquad 第一作者}{\textbf{2023.12\textasciitilde \ 2024.1}}
    \begin{itemize}
    \item[\faThumbTack] \kaishu 任务创新: 提出了一种新的文本友好型文本到图像生成任务,该任务创建的图像既满足提示,又为预定义的文本位置保留空间。基准测试包括专门的数据集和量身定制的评估指标。
        % \item[\faThumbTack] \kaishu 我们的即插即用的方法在T2I模型中采用力导向的注意力引导来生成图像,这些图像在战略上为预定义的文本区域保留空白,甚至为黄金比例的文本或图标保留空白。通过观察交叉注意力Map如何影响物体放置,我们使用力导向的方法检测和移动冲突对象,结合空间排除交叉注意力约束,在空白区域中平滑注意力。
        \item[\faThumbTack] \kaishu 算法创新: 提出一个即插即用,无需训练的diffusion管道。这种方法允许在图像中用户指定的位置动态放置文本,并生成或调整图像以适应文本。主要模块为力导向交叉注意引导。
        % \item[\faThumbTack] \kaishu 基于当前主流的基于注意力的图像编辑工作, 完成Diffusers的Pipeline核心代码, 撰写论文。
        % \item[\faThumbTack] \kaishu 该模块战略性地指导图像生成过程中的交叉关注地图,确保文本和图像的和谐布局。实践发现注意力结合SegmentAnythin很适合用来提高跨模态的VLM视觉理解。 
    \end{itemize}
%-------------------------------------------------------------------------------
\datedsubsection{\textbf{NeuralBionicSyn: Out-of-distribution 3D Shape Synthesis via Implicit Representation \linebreak for Organic Biologically Inspired Design}}{\textbf{2021.9\textasciitilde \ 2023.10}}
\role {期刊论文TVCG(在投)}{第一作者}
    \begin{itemize}
    \item[\faThumbTack] \kaishu 任务创新: 不同于传统生成任务,该任务专注于训练数据分布外区域的融合语义的生成问题。
    \item[\faThumbTack] \kaishu 算法创新: 提出创意生成物求解器来解决分布外生成的黑盒函数多峰优化问题,并理论上证明了生成式仿生设计与分布外生成高度相关。
    \item[\faThumbTack] \kaishu 系统创新: 系统允许对设计变体进行高阶的探索,从而防止由于过多的生成物候选对象导致用户不感兴趣。
        % \item[\faThumbTack] \kaishu 提出了神经三维仿生设计算法。将创意仿生设计问题视为基于参考数据集的隐空间分布外生成问题。将生成具有仿生价值的创新形状作为优化问题,在分布外区域进行约束探索,这一过程依赖于混合评分函数以筛选出候选形状集合。研究引入了人在回路的局部流形子空间探索技术,从而使得设计师能够高效地发掘和迭代设计变体。
        % \item[\faThumbTack] \kaishu 独立完成隐式编解码网络模型训练, 生成物求解器运算, Tornado后端, Blender渲染。 合作完成交互式Vue前端的系统实现。
        % \item[\faThumbTack] \kaishu 黑盒的分布外生成可以扩展到其他具有表示空间生成任务, 提取出多样性的局部最优解来得到多样的高质量生成候选。 这段gap经历也启发我改变研究模式, 进行2月一篇的研究转型。
    \end{itemize}
%-------------------------------------------------------------------------------
% \section{\hspace{0.25em}\makebox[0.75em][c]{\faPaste} \fangsong\textbf{毕业设计}}
% \datedsubsection{\textbf{生成式智能三维仿生设计研究}}{\textbf{2024.2 \textasciitilde \ 2024.3}}
%     \begin{itemize}
%         \item[\faThumbTack]  \kaishu 和仿生设计研究项目内容一致,补充了工程细节。
%         \item[\faThumbTack]  \kaishu 引入新一套工具链进行论文加速, 详情见知乎博客:LLM时代的计算机科学毕业论文撰写小记录。
%         % \item[\faThumbTack]  \kaishu 复制相应文字到此处
%     \end{itemize}
%-------------------------------------------------------------------------------
%-------------------------------------------------------------------------------
% \section{\hspace{0.25em}\makebox[0.75em][c]{\faPaperPlane} \fangsong\textbf{校园实践}}
% \datedsubsection{\textbf{\LaTeX 专业排版工程学院保卫科}\qquad \textit{保安大队长}}
% {\textbf{2020.9 \textasciitilde \ 2022.12}}
% \begin{itemize}
%     \item[\faThumbTack]  \kaishu 负责保卫工作;负责保卫大学生;负责保卫大领导
% \end{itemize}
%-------------------------------------------------------------------------------
% \datedsubsection{\textbf{\LaTeX 专业排版工程学院后勤部}\qquad \textit{后勤部部长}}
% {\textbf{2020.9 \textasciitilde \ 2022.3}}
%     \begin{itemize}
%         \item[\faThumbTack]  \kaishu 负责后勤工作;负责喂饱大学生;负责喂饱大领导
%     \end{itemize}
%-------------------------------------------------------------------------------
%-------------------------------------------------------------------------------
% \section{\hspace{0.25em}\makebox[0.75em][c]{\faHeartbeat} \fangsong\textbf{志愿经历}}
% \datedsubsection{\textbf{\LaTeX 专业排版工程学院青年志愿部}\qquad \textit{部长}}
% {\textbf{202×.×× \textasciitilde \ 202×.××}}
%     \begin{itemize}
%         \item[\faThumbTack]  \kaishu 负责组织志愿活动
%     \end{itemize}
%-------------------------------------------------------------------------------
%-------------------------------------------------------------------------------
\section{\hspace{0.25em}\makebox[0.75em][c]{\faTrophy} \fangsong\textbf{荣誉奖项}}
    \vspace{0.1em}
    \begin{itemize}
        \item \datedline{\kaishu 上海市优秀毕业生\&优秀论文}{\textbf{2021.7}}
        \item \datedline{\kaishu 全国大学生数学建模竞赛\qquad \qquad \textbf{上海赛区本科组一等奖}}{\textbf{2019.12}}
\item \datedline{\kaishu 上海市奖学金 \qquad \textbf{2018.12}}{国家奖学金 \qquad  \textbf{2019.12}}
    \end{itemize}
%-------------------------------------------------------------------------------
%-------------------------------------------------------------------------------
\section{\hspace{0.25em}\makebox[0.75em][c]{\faPuzzlePiece} \fangsong\textbf{专业技能}}
\noindent
    \begin{minipage}[t]{0.75\textwidth}
     \raggedright
        \begin{itemize}
            \item \kaishu\textbf{算法}: PyTorch, Diffusers (熟练), Transformers \textbf{后端}: FastAPI (Agents开发)\linebreak\textbf{前端}: Vue React Three.js (Demo交互) \textbf{运维}:Docker(熟练, 管理员) 
        \end{itemize}
    \end{minipage}%
    \begin{minipage}[t]{0.5\textwidth}
        \begin{itemize}
            \item \kaishu\textbf{英语能力}: CET-6 (562)
            % , IELTS(9.0), TOEFL(115)
        \end{itemize}
    \end{minipage}
%-------------------------------------------------------------------------------
%-------------------------------------------------------------------------------
% \section{\hspace{0.25em}\makebox[0.75em][c]{\faSignal} \fangsong\textbf{硕士规划}}
% \vspace{0.1em}
%     \begin{itemize}
%         \item \kaishu 具体规划,复制相应文字到此处;复制相应文字到此处;复制相应文字到此处
%     \end{itemize}
%-------------------------------------------------------------------------------
%-------------------------------------------------------------------------------
\end{document}